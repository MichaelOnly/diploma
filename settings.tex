

\onehalfspacing

\usepackage{float}
\usepackage{listings}  
\usepackage{graphicx}
\usepackage{color}
\usepackage{caption}
\usepackage{fancyhdr}
\graphicspath{{./images/}}

% Параметры названия для иллюстраций (Рисунок 1 - <название>)
\DeclareCaptionLabelFormat{PictureCaptionFormat}{Рисунок {#2}}
\captionsetup[figure] {
labelformat=PictureCaptionFormat,
skip=0pt,
format=hang,
justification=raggedright,
labelsep=endash
}

% Параметры названия для таблиц (Таблица 1 - <название>)
\DeclareCaptionLabelFormat{TableCaptionFormat}{Таблица {#2}}
\captionsetup[table] {
labelformat=TableCaptionFormat,
skip=0pt,
format=hang,
justification=raggedright,
singlelinecheck=off,
labelsep=endash
}

\DeclareCaptionLabelFormat{ListingCaptionFormat}{Листинг {#2}}
\captionsetup[lstinputlisting] {
  labelformat=PictureCaptionFormat,
  skip=0pt,
  format=hang,
  justification=raggedright,
  labelsep=endash
}

%\setcounter{tocdepth}{2}

\definecolor{mygreen}{rgb}{0,0.6,0}
\definecolor{mygray}{rgb}{0.5,0.5,0.5}
\definecolor{mymauve}{rgb}{0.58,0,0.82}

\lstloadlanguages{Python}
\lstset{
  language=Python,
  keywordstyle={\bfseries \color{blue}},
  captionpos=b,
  tabsize=4,
  basicstyle=\footnotesize,
  breaklines=true,
  breakatwhitespace=true,
  extendedchars=true,
  backgroundcolor=\color{white},
  commentstyle=\color{mygreen},
  showstringspaces=false,
  escapeinside={\%*}{*)},
  stringstyle=\color{mymauve},
  numbers=left,
  numberstyle=\tiny\color{gray}
  }