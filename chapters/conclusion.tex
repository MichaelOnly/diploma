\begin{itemize}
   \item Изучены методы генерации и аугментирования данных, методы представления текстовых данных в векторном виде, методы машинного обучения для классификации текста в лице нейронных сетей, архитектуры моделей, наиболее часто используемых для обработки естественного языка и классификации текста в частности, программные средства для обучения и тестирования этих моделей, для логирования данных во время обучения и их анализа;
   \item Разработаны программы для генерации данных и проведения исследований;
   \item Создан новый набор данных;
   \item Проведены исследования для поиска оптимальных параметров для каждой выбранной модели и сравнения моделей между собой;
   \item Получена основная часть NLU модуля, а именно модель для классификации намерений. Наиболее оптимальной для этой задачи архитектурой является RoBERTa;
   \item Не самое высокое качество обучающего набора данных, что связано со сложностью выделения некоторых интентов и несовершенством искусственно сгенерированных и аугментированных данных.
\end{itemize}

%Таким образом, была получена основная часть NLU модуля, а именно модель для классификации намерений. Наиболее оптимальной для этой задачи архитектурой является RoBERTa. Для получения итоговой модели был создан новый набор данных, а также были проведены иследования для поиска оптимальных параметров для каждой выбранной модели и для сравнения моделей между собой. Для этого были изучены методы генерации и аугментирования данных, методы представления текстовых данных в векторном виде, методы машинного обучения для классификации текста в лице нейронных сетей, а также архитектуры моделей, наиболее часто используемых для обработки естественного языка и классификации текста в частности. Кроме того, были изучены программные средства для обучения, тестирования этих моделей, логгирования данных во время обучения и их анализа, а также разработаны программы для генерации данных и проведения исследования.

%Однако по результатам тестирования модели, можно сделать вывод о не самом высоком качестве обучающего набора данных, что связано со сложностью выделения некоторых интентов и несовершенством искусственно сгенерированных и аугментированных данных, что требует дальнешей работы в этом направлении.