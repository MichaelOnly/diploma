Модуль понимания естественного языка (Natural Language Understanding или NLU) является важным компонентом многих приложений обработки 
естественного языка (Natural Language Processing или NLP), включая чат-ботов и виртуальных помощников. Его основная функция заключается в том, чтобы  позволить системе понимать и интерпретировать ввод человеческого языка. Основной частью NLU модуля является модуль классификации намерений (интентов). Его функция заключается в том, чтобы точно определить основное намерение, стоящее за вводом пользователей на естественном языке. В данной работе такой модуль создаётся для системы создания чат-агентов для игр. Его функция в этой системе заключается в понимании цели запроса игрока и формировании более полного и корректного ответа на этот запрос, что позволяет улучшить опыт игрока за счёт более полного взаимодействия с игрой. В качестве основного для всей системы был выбран английский язык, что связано с большим количеством доступных моделей и данных. 

Эффективность модуля классификации намерений зависит от двух основных факторов: модель и обучающий набор данных (датасет). Модели имеют различные архитектуру, размер и метод предобучения, что оказывает серьёзное влияние на их способность решать те или иные задачи. Кроме того, разные модели потребляют разное количество ресурсов (видеопамять, время и т.д.) при обучении и использовании. Основным и самым распространённым семейством архитектур в NLP является класс моделей трансформеры. В связи с большим количеством моделей, относящихся к этому семейству, было принято решение выбрать несколько наиболее популярных моделей и провести их сравнение при дообучении и решении задачи выделения намерений.

Основными параметрами датасета, влияющими на качество итоговой системы, являются его объём и разнообразие примеров (т.е. количество покрываемых ситуаций). Помимо всего прочего, для обучения модели выделения намерений, входящей в состав NLU модуля системы создания чат-агентов для игр, 
требуется специфический набор данных. Главное отличие такого набора в содержании специфических для игровой индустрии классов намерений, 
предполагающих более полное взаимодействие игрока и неигровых персонажей, из-за чего найти подходящие чистые данные 
в открытом доступе достаточно сложно. В связи с этим, для разработки системы классификации намерений требуется создание 
нового набора данных.