В качестве языка программирования для написания программ генерации данных и обучения моделей используется \texttt{Python}.
Для генерации данных использовались стандартные средства языка, а для их сохранения использовалась библиотека \texttt{Pandas}. Для использования и обучения моделей, для проведения экспериментов использовались специализированные библотеки: \texttt{Transformers}, \texttt{Datasets}, \texttt{WandB}, \texttt{Evaluate}.

\texttt{Transformers} используется для загрузки, хранения в облаке, использования и обучения моделей класса Transformer. Библиотека разработана 
HuggingFace и связана с открытым облачным хранилищем моделей HuggingFace. 

Ещё одной библиотекой от HuggingFace является \texttt{Datasets}. Эта библиотека используется для загрузки, хранения в облаке и использования 
наборов данных и связанных с ними метаданных. Кроме того, она связана с библиотекой \texttt{Transformers}, что позволяет быстро и просто 
использовать наборы данных из \texttt{Datasets} для обучения моделей с помощью \texttt{Transformers}.

Библиотека \texttt{Evaluate} так же разработана HuggingFace и используется для высчитывания различных метрик при обучении и тестировании моделей.

Библиотека \texttt{WandB} разработана Weights\&Biases. Используется для логирования процесса обучения, хранения моделей, наборов данных 
и визуализации метрик и других системных показателей во время обучения моделей. \texttt{WandB} также тесно связана с библиотекой 
\texttt{Transformers}. Кроме того, функционал библиотеки \texttt{WandB} позволяет дообучать несколько моделей с различными 
параметрами последовательно для проведении экспериментов, связанных с различными параметрами обучения моделей.

\section{ПРОГРАММА ДЛЯ ГЕНЕРАЦИИ ДАННЫХ}
В программе для создания набора данных были написана функция для создания примеров методом маскирования, которая принимает на вход список шаблонов, список сущностей для подстановки в шаблоны и маску, вместо которой подставляются сущности. Также была написана функция для перефразирования примеров с помощью соответсвущей модели, которая принимает на вход список примеров для перефразирования. Создание примеров вынесено в отдельные функции для каждого класса, после чего данные сохранялись в соответствущих файлах отдельно. Текст основных функций и пример генерации данных по одному из классов можно найти в приложении \ref{app:data-generating}.

\section{ПРОГРАММА ДЛЯ ПОИСКА ОПТИМАЛЬНЫХ ПАРАМЕТРОВ}
Для проведения исследования была разработана программа на языке \texttt{Python} с использованием описанных ранее библиотек. Программа 
загружает подготовленный набор данных и токенизирует его для дообучения модели. Далее с помощью функций класса из библиотеки \texttt{WandB} 
\texttt{sweep} обучается несколько моделей с различными параметрами обучения. Данные о процессе обучения и валидации во время обучения 
загружаются на сайт Weights\&Biases, и по ним автоматически строятся графики для анализа. Текст программы для исследования можно 
найти в приложении \ref{app:example}.

\section{ПРОГРАММА ДЛЯ СРАВНЕНИЯ МОДЕЛЕЙ}
Для сравнения моделей была разработана программа для обучения и тестирования моделей, а также логирования данных во время обучения 
с использованием описанных ранее библиотек. Программа состоит из 4-х блоков:
\begin{enumerate}
   \item Загрузка и предподготовка данных для обучения и тестирования;
   \item Загрузка модели для обучения и тестирования;
   \item Формирование параметров для обучения моделей;
   \item Обучение и тестирование.
\end{enumerate}

Данные о процессе обучения и валидации во время обучения загружаются на сайт Weights\&Biases и по ним автоматически строятся графики для анализа. Текст программы для исследования можно найти в приложении \ref{app:train}.