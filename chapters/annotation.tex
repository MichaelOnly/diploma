Отчёт 60 с., 5 ч., 17 рис., 3 табл., 23 источника, 5 прил.


МАШИННОЕ ОБУЧЕНИЕ, НЕЙРОННЫЕ СЕТИ, ОБУЧЕНИЕ С УЧИТЕЛЕМ, ПОНИМАНИЕ ЕСТЕСТВЕННОГО ЯЗЫКА, КЛАССИФИКАЦИЯ ТЕКСТА, КЛАССИФИКАЦИЯ НАМЕРЕНИЙ, ТРАНСФОРМЕРЫ


В этой работе анализируется эффективность различных моделей архитектуры трансформер в классификации намерений и представлена разработка модуля понимания естественного языка на основе нейронных сетей архитектуры трансформер для системы создания диалоговых моделей. Модуль, разработанный в этом исследовании, сосредоточен на классификации намерений и составляет основную часть модуля понимания естественного языка для системы.

По теме работы было выступление на XVI Всероссийской научной конференции молодых ученых <<НАУКА. ТЕХНОЛОГИИ. ИННОВАЦИИ>> и была опубликована статья \cite{paper}.

